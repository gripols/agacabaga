\documentclass{article}

\usepackage{fancyhdr}
\usepackage{extramarks}
\usepackage{amsmath}
\usepackage{amsthm}
\usepackage{amsfonts}
\usepackage{tikz}


\usetikzlibrary{automata,positioning}

%
% Basic Document Settings
%

\topmargin=-0.45in
\evensidemargin=0in
\oddsidemargin=0in
\textwidth=6.5in
\textheight=9.0in
\headsep=0.25in

\linespread{1.1}

\pagestyle{fancy}
\lhead{\hmwkAuthorName}
\chead{\hmwkClass\ (\hmwkClassInstructor\ \hmwkClassTime): \hmwkTitle}
\rhead{\firstxmark}
\lfoot{\lastxmark}
\cfoot{\thepage}

\renewcommand\headrulewidth{0.4pt}
\renewcommand\footrulewidth{0.4pt}

\setlength\parindent{0pt}

%
% Create Problem Sections
%

\newcommand{\enterProblemHeader}[1]{
    \nobreak\extramarks{}{Problem \arabic{#1} continued on the next page\ldots}\nobreak{}
    \nobreak\extramarks{Problem \arabic{#1} (continued)}{Problem \arabic{#1} continued on the next page\ldots}\nobreak{}
}

\newcommand{\exitProblemHeader}[1]{
    \nobreak\extramarks{Problem \arabic{#1} (continued)}{Problem \arabic{#1} continued on the next page\ldots}\nobreak{}
    \stepcounter{#1}
    \nobreak\extramarks{Problem \arabic{#1}}{}\nobreak{}
}

\setcounter{secnumdepth}{0}
\newcounter{partCounter}
\newcounter{homeworkProblemCounter}
\setcounter{homeworkProblemCounter}{1}
\nobreak\extramarks{Problem \arabic{homeworkProblemCounter}}{}\nobreak{}

\newenvironment{homeworkProblem}[1][-1]{
    \ifnum#1>0
        \setcounter{homeworkProblemCounter}{#1}
    \fi
    \section{Problem \arabic{homeworkProblemCounter}}
    \setcounter{partCounter}{1}
    \enterProblemHeader{homeworkProblemCounter}
}{
    \exitProblemHeader{homeworkProblemCounter}
}

%
% Homework Details
%

\newcommand{\hmwkTitle}{Assignment\ B}
\newcommand{\hmwkDueDate}{March 19, 2024}
\newcommand{\hmwkClass}{MCV4U}
\newcommand{\hmwkClassTime}{Unit 2}
\newcommand{\hmwkClassInstructor}{M. Ellenbogen}
\newcommand{\hmwkAuthorName}{\textbf{G. Polstvin}}

%
% Title Page
%

\title{
    \vspace{2in}
    \textmd{\textbf{\hmwkClass:\ \hmwkTitle}}\\
    \normalsize\vspace{0.1in}\small{Due\ on\ \hmwkDueDate\ at 6:00pm}\\
    \vspace{0.1in}\large{\textit{\hmwkClassInstructor\ \hmwkClassTime}}
    \vspace{3in}
}

\author{\hmwkAuthorName}
\date{}

\renewcommand{\part}[1]{\textbf{\large Part \Alph{partCounter}}\stepcounter{partCounter}\\}

% For derivatives
\newcommand{\deriv}[1]{\frac{\mathrm{d}}{\mathrm{d}x} (#1)}

% For partial derivatives
\newcommand{\pderiv}[2]{\frac{\partial}{\partial #1} (#2)}

% Integral dx
\newcommand{\dx}{\mathrm{d}x}

% Alias for the Solution section header
\newcommand{\solution}{\textbf{\large Solution}}

\begin{document}

\maketitle

\pagebreak

% Begin Homework.

% PROBLEM 1
% PROBLEM 1
% PROBLEM 1

\begin{homeworkProblem}
    Evaluate and state what this limit tells you about the function.
    \[ \lim_{{x \to 0^+}} \frac{{x-5}}{{x(x+1)}} \]

    \textbf{Solution} \\

    The limit of $\frac{x-5}{x(x+1)}$ as $x$ approaches 0 does not exist. \\


    \textbf{Calculations} \\

    By definition, the limit exists at a point only if the left-hand limit and the right-hand limit must be equal, so we need:

    \[
        \lim_{x \to 0^-} f(x) = \lim_{x \to 0^+} f(x)
    \] \\

    We have already shown that the right-hand limit is:

    \[
        \lim_{x \to 0^+} f(x) = \lim_{x \to 0^+} \frac{x-5}{x(x+1)} = -\infty
    \] \\

    Evaluating the left hand limit produces:

    \[
        \begin{split}
            \lim_{x \to 0^-} f(x) &= \lim_{x \to 0^-} \frac{x-5}{x(x+1)} \\
            &= \lim_{x \to 0^-} \left(\frac{1}{\frac{1}{x-5} - \frac{1}{x}}\right) \\
            &= \lim_{x \to 0^-} \left(\frac{1}{\frac{-5}{x(x-5)}}\right) \\
            &= \lim_{x \to 0^-} \left(-\frac{x(x-5)}{5}\right) \\
            &= +\infty
        \end{split}
    \] \\

    Since the left-hand limit is $+\infty$ and the right-hand limit is $-\infty$, the limits are not equal. \\

    Therefore, the limit of $\frac{x-5}{x(x+1)}$ as $x$ approaches 0 does not exist.

\end{homeworkProblem}

\pagebreak

% PROBLEM 2
% PROBLEM 2
% PROBLEM 2

% FINISH THIS SHIT
% RATIONAL NUMBERS ONLY
% STATE SECOND DERIVATIVE VALUES @ THE EXTREMA

\pagebreak

\begin{homeworkProblem}
    For the function $f(x) = \frac{1}{x(x-1)^2}$, find the local extrema.
    Then, classify the local extrema using the second derivatives test. \\

    \textbf{Solution}
    \[
        \text{Local Extrema: }\left(\frac{1}{3}, \frac{27}{4}\right)
    \] \\

    \textbf{Calculations} \\

    First derivative: \\

    \begin{align*}
        f'(x) & = \frac{(x(x-1)^2)' \cdot 1 - 1 \cdot (x(x-1)^2)'}{(x(x-1)^2)^2}        \\
              & = \frac{((x-1)^2 + 2x(x-1)) \cdot 1 - 1 \cdot (x(x-1)^2)}{(x(x-1)^2)^2} \\
              & = \frac{(x^2 - 2x + 1 + 2x^2 - 2x)}{x^2(x-1)^4}                         \\
              & = \frac{(3x^2 - 4x + 1)}{x^2(x-1)^4}                                    \\
              & = -\frac{3x-1}{(x-1)^3x^2}
    \end{align*} \\

    Second derivative: \\

    \begin{align*}
        f''(x) & = \frac{(x^2(x-1)^4)' \cdot (3x-1) - (x^2(x-1)^4) \cdot (3x-1)'}{(x^2(x-1)^4)^2}                 \\
               & = \frac{((x^2(x-1)^4)' \cdot (3x-1) - (x^2(x-1)^4) \cdot 3)}{x^4(x-1)^8}                         \\
               & = \frac{((4x^3(x-1)^3 + x^2(x-1)^4) \cdot (3x-1) - x^2(x-1)^4 \cdot 3)}{x^4(x-1)^8}              \\
               & = \frac{(12x^4(x-1)^3 - 4x^3(x-1)^3 + x^2(x-1)^4) \cdot (3x-1) - x^2(x-1)^4 \cdot 3}{x^4(x-1)^8} \\
               & = \frac{(12x^4(x-1)^3 - 4x^3(x-1)^3 + x^2(x-1)^4) \cdot (3x-1) - 3x^2(x-1)^4}{x^4(x-1)^8}        \\
               & = \frac{(36x^5 - 108x^4 + 108x^3 - 36x^2) \cdot (x-1)^3 - 3x^2(x-1)^4}{x^4(x-1)^8}               \\
               & = \frac{2(6x^2-4x+1)}{(x-1)^4x^3}
    \end{align*} \\

    Set the first derivative equal to zero, then the numerator, solve for $x$, and divide by three.
    \[
        -\frac{3x-1}{(x-1)^3x^2} = 0
    \]
    \[
        3x - 1 = 0
    \]
    \[
        3x = 1
    \]
    \[
        x = \frac{1}{3}
    \]

    Evaluate the second derivative with $x$ equal to $\frac{1}{3}$.
    \[
        f''\left(\frac{1}{3}\right) = \frac{2\left(6\left(\frac{1}{3}\right)^2-4\left(\frac{1}{3}\right)+1\right)}{\left(\frac{1}{3}-1\right)^4\left(\frac{1}{3}\right)^3}
    \]
    \[
        f''\left(\frac{1}{3}\right) = \frac{2\left(\frac{2}{9}-\frac{4}{9}+1\right)}{(-\frac{2}{3})^4\left(\frac{1}{27}\right)}
    \]
    \[
        f''\left(\frac{1}{3}\right) = \frac{2\left(\frac{7}{9}\right)}{\frac{16}{81}}
    \]
    \[
        f''\left(\frac{1}{3}\right) = \frac{729}{8}
    \]

    Since $f''\left(\frac{1}{3}\right) > 0$, the point $\left(\frac{1}{3}, f\left(\frac{1}{3}\right)\right)$ is a local maximum.

    Now substitute $x = \frac{1}{3}$ into the function $f(x)$ to find the $y$-value.
    \[
        f\left(\frac{1}{3}\right) = \frac{1}{\left(\frac{1}{3}\right)\left(\frac{1}{3}-1\right)^2}
    \]
    \[
        f\left(\frac{1}{3}\right) = \frac{1}{\left(\frac{1}{3}\right)\left(-\frac{2}{3}\right)^2}
    \]
    \[
        f\left(\frac{1}{3}\right) = \frac{27}{4}
    \] \\

    Therefore, the local maximum of $f(x) = \frac{1}{x(x-1)^2}$ is $\left(\frac{1}{3}, \frac{27}{4}\right)$.
\end{homeworkProblem}

\pagebreak

% PROBLEM 3
% PROBLEM 3
% PROBLEM 3

% WHEN WRITING QUIZ, PUT ANSWER IN BRACKET INTERVAL NOTATION.

\begin{homeworkProblem}
    The function $f(x) = x^3 - 6x^2 + 9x$ is concave down on which interval? \\

    \textbf{Solution}
    The function $f(x) = x^3 - 6x^2 + 9x$ is concave downward on $-\infty < x < 2$. \\

    \textbf{Calculations}
    Begin by taking the first derivative:
    \[
        \begin{aligned}
            \frac{\mathrm{d}}{\mathrm{d}x}(x^3-6x^2+9x)
              & = 3x^2 - 12x + 9     \\
              & = 3x^2-12x+9         \\
              & = f'(x) = 3x^2-12x+9
        \end{aligned}
    \]

    Now, take the second derivative:
    \[
        \begin{aligned}
            \deriv{3x^2-12x+9 }
              & = 3 \cdot \deriv{x^2 - 12} \cdot \deriv{x} + \deriv{9} \\
              & = 3 \cdot 2x - 12 \cdot 1 + 0                          \\
              & = 6x - 12                                              \\
              & = f''(x) = 6x-12
        \end{aligned}
    \]

    To find the points of inflection, we set $x$ in the second derivative to $0$, which produces:
    \[
        \begin{aligned}
            6x-12 & = 0  \\
            6x    & = 12 \\
            x     & = 2
        \end{aligned}
    \]

    From here, we plug it into our original equation:
    \[
        \begin{aligned}
            f(x) = x^3 - 6x^2 + 9x \\
              & = (2)^3 - 6(2)^2 + 9(2) \\
              & = 2
        \end{aligned}
    \]

    Which makes the point of inflection for $f(x) = x^3 - 6x^2 + 9x$ equal to
    \((2, 2)\).

    To find the interval, we can plug in another number for $x$ and calculate it. Let's use 4:

    \[
        \begin{aligned}
            f(x) = x^3 - 6x^2 + 9x \\
              & = (4)^3 - 6(4)^2 + 9(4) \\
              & = 4
        \end{aligned}
    \]

    This gives us a co-ordinate of $(4, 4)$. We will assume this interval is concave upward $(2<x<\infty \:)$, which makes $-\infty \:<x<2$ the concave downward interval. \\

    \textbf{Proof of Concavity} \\

    To prove that the interval $-\infty < x < 2$ is concave downward, we need to show that the second derivative is negative for this interval. \\

    Let $x = 1$,

    \[
        \begin{aligned}
            f''(x) & = 6x - 12   \\
                   & = 6(1) - 12 \\
                   & = -6
        \end{aligned}
    \] \\

    Since $f''(1) < 0$, the function is concave downward on the interval $-\infty < x < 2$.

\end{homeworkProblem}

% PROBLEM 4
% PROBLEM 4
% PROBLEM 4

\begin{homeworkProblem}

    Consider the function \( f(x) =-\frac{x^2-1}{-3(x+4)(x-2} \)

    \begin{enumerate}
        \item To use a limit and find its horizontal asymptote, what must be done first?
        \item Find the horizontal asymptote as \(x \to -\infty \) using your limit.
    \end{enumerate} 
        
    \textbf{Solution} \\

    \textbf{Calculations} \\

\end{homeworkProblem}

\pagebreak

% PROBLEM 5
% PROBLEM 5
% PROBLEM 5

\begin{homeworkProblem}

    For the function \( f(x) = 3x^4 - 12x^3 \), find the points of inflection. \\

    \textbf{Solution} \\

    Inflection points are found at \((0, 0)\), \((2, -48)\) \\

    \textbf{Calculations} \\

    First derivative:

    \[
        \begin{split}
            f'(x) &= \frac{d}{dx}(3x^4-12x^3) \\
            &= 3 \cdot \frac{d}{dx}(x^4) - 12 \cdot \frac{d}{dx}(x^3) \\
            &= 3 \cdot 4x^3 - 12 \cdot 3x^2 \\
            &= 12x^3 - 36x^2 \\
            &= 12x^2(x-3)
        \end{split}
    \]


    Second derivative:
    \[
        \begin{split}
            f''(x) &= \frac{d^2}{dx^2}(12x^2(x-3)) \\
            &= 12 \cdot \frac{d}{dx}(x^2(x-3)) \\
            &= 12 \left(2x(x-3) + x^2\right) \\
            &= 12(2x^2 - 6x + x^2) \\
            &= 12(3x^2 - 6x)
        \end{split}
    \]

    Which can be further simplified as:

    \[
        \begin{split}
            f''(x) &= 36x(x-2)
        \end{split}
    \]

    Third derivative:

    \[
        \begin{split}
            f'''(x) &= \frac{d^3}{dx^3}(36x(x-2)) \\
            &= 36 \cdot \frac{d}{dx}(x(x-2)) \\
            &= 36(1 \cdot (x-2) + x) \\
            &= 36(x-2+x) \\
            &= 72x-72
        \end{split}
    \]

    Setting the second derivative equal to zero, we get:
    \[
        36x(x-2) = 0 \implies x = 0, 2
    \] \\

    To check if these points are points of inflection, we need to evaluate the sign of the third derivative at these points.

    For $x = 0$:
    \[
        f'''(0) = 72(0) - 72 = -72 < 0
    \]

    For $x = 2$:
    \[
        f'''(2) = 72(2) - 72 = 72 > 0
    \]

    Since the third derivative changes sign from negative to positive as $x$ increases through $x = 0$ and $x = 2$, these points are indeed points of inflection. \\

    To find the corresponding $y$-values, we substitute $x = 0$ and $x = 2$ into the original function:
    \[
        f(0) = 3(0)^4 - 12(0)^3 = 0 \quad \text{and} \quad f(2) = 3(2)^4 - 12(2)^3 = -48
    \]

    Therefore, the points of inflection are \((0, 0)\) and \((2, -48)\).


\end{homeworkProblem}

\end{document}
